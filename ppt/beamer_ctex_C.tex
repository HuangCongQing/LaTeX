%
% 使用 xelatex 编译
%
\documentclass[10pt,compress,t,noamsthm,notheorem,handout,table]{ctexbeamer}
\useoutertheme[subsection=false]{miniframes}
\setbeamercolor{separation line}{use=structure,bg=structure.fg!50!bg}
\useinnertheme{circles}
\usefonttheme[onlymath]{serif}
\setbeamertemplate{navigation symbols}{}
\setbeamersize{text margin left=0.5cm, text margin right=0.5cm}
\setbeamerfont{frametitle}{size=\large}
\definecolor{myfoot}{rgb}{0.5,0.2,0.5}
\setbeamertemplate{footline}% 自定义页脚
  { \leavevmode\mbox{%
    \begin{beamercolorbox}[wd=.75\paperwidth,ht=2.25ex,dp=1ex,left]{myfootline}%
        \rule{2em}{0pt}\color{myfoot}\ttfamily\scriptsize
    \end{beamercolorbox}%
    \begin{beamercolorbox}[wd=.25\paperwidth,ht=2.25ex,dp=1ex,right]{myfootline}%
       {\color{myfoot}\ttfamily\scriptsize\insertframenumber{}/%
        \inserttotalframenumber\hspace*{3ex}}
    \end{beamercolorbox}}
    \vskip0pt }

%%%%% ===== 常用宏包 =======================================================
\usepackage{amsmath,amssymb,amsfonts,bm}
\usepackage{graphicx}
  \graphicspath{{figures/}}
  \definecolor{darkblue}{rgb}{0.1,0,0.85}
  \hypersetup{pdfborder=001,colorlinks=true,linkcolor=darkblue,urlcolor=blue}
\usepackage{bbding}

%%%%% ===== 自定义列表 ======================================================
\newcommand{\Bullet}{{\fontsize{6pt}{6pt}\selectfont\CircleSolid}}
\newcommand{\Hand}{{\fontsize{8pt}{6pt}\selectfont\HandRight}}
\newcommand{\zhu}{{\color{blue!40}\Bullet}}
\newcommand{\zhuu}{{\color{red!80}\Hand}}
\newcommand{\labeli}{\zhu}
\newenvironment{myitem}
  {\begin{list}{{\hfill\raisebox{0pt}{\labeli}}}{%
    \setlength{\leftmargin}{1.2em}\labelwidth0.8em\labelsep.4em%
    \itemsep1ex\parsep2pt\itemindent0pt\topsep0pt}}{\end{list}}
\newenvironment{subitem}
  {\begin{list}{{\hfill\raisebox{0pt}{\zhuu}}}{%
    \setlength{\leftmargin}{1.2em}\labelwidth0.8em\labelsep.4em%
    \itemsep0ex\parsep2pt\itemindent0pt\topsep0pt}}{\end{list}}
\usepackage{colortbl}
\usepackage{booktabs}
\usepackage{tikz}
\usetikzlibrary{arrows}
\usepackage[framemethod=tikz]{mdframed}

%%%%% ===== 定理环境 ======================================================
\usepackage[amsmath,thref,thmmarks,hyperref]{ntheorem}
\theorempreskipamount1.2em  % spacing before the environment
\theorempostskipamount0em % spacing after the environment
%\theorempostwork{\noindent}
\theoremstyle{nonumberplain}%{nonumberbreak}
\theoremheaderfont{\color{blue}\upshape}
%\theorembodyfont{\kaishu\color{black}}
\theoremindent0em
\theoremseparator{:\hspace{0.2em}}
%\theoremnumbering{arabic}
\colorlet{thmcolor}{gray!40}
\newmdtheoremenv[linecolor=thmcolor,middlelinewidth=1pt,
    roundcorner=3pt,backgroundcolor=white,%
    innertopmargin=0.5em,innerbottommargin=0.5em,%
    innerleftmargin=3pt,innerrightmargin=3pt,%
    skipbelow=0.5em,skipabove=1em,%
    splittopskip=\topskip,ntheorem]{theorem}%
    {定理}
\newmdtheoremenv[linecolor=thmcolor,middlelinewidth=0.5pt,
    roundcorner=3pt,backgroundcolor=white,%
    innertopmargin=0.5em,innerbottommargin=0.5em,%
    innerleftmargin=3pt,innerrightmargin=3pt,%
    skipbelow=0.5em,skipabove=1em,%
    splittopskip=\topskip,ntheorem]{corollary}%
    {推论}
\newmdtheoremenv[linecolor=thmcolor,middlelinewidth=0.5pt,
    roundcorner=3pt,backgroundcolor=white,%
    innertopmargin=0.5em,innerbottommargin=0.5em,%
    innerleftmargin=3pt,innerrightmargin=3pt,%
    skipbelow=0.5em,skipabove=1em,%
    splittopskip=\topskip,ntheorem]{lemma}%
    {引理}
%\newmdtheoremenv[linecolor=thmcolor,middlelinewidth=0.5pt,
%    roundcorner=3pt,backgroundcolor=white,%
%    innertopmargin=0.5em,innerbottommargin=0.5em,%
%    innerleftmargin=3pt,innerrightmargin=3pt,%
%    skipbelow=0.5em,skipabove=1em,%
%    splittopskip=\topskip,ntheorem]{definition}%
%    {定义}
\newmdenv[innertopmargin=0pt,roundcorner=5pt,linewidth=2pt,
    linecolor=gray!40,
    innertopmargin=1.2em,innerleftmargin=1ex,innerrightmargin=1ex,
    singleextra={
      \node[anchor=west,xshift=1em,fill=gray!10,rounded corners=4pt,%
       draw=green!50,line width=1pt]%
       at (P-|O) {\ \color{blue} 定\ 义\ \mbox{}};
    }]{definition}

\theoremstyle{plain}
\newmdtheoremenv[linecolor=thmcolor,middlelinewidth=0.5pt,
    roundcorner=3pt,backgroundcolor=white,%
    innertopmargin=0.5em,innerbottommargin=0.5em,%
    innerleftmargin=3pt,innerrightmargin=3pt,%
    skipbelow=0.5em,skipabove=1em,%
    splittopskip=\topskip,ntheorem]{example}%
    {例}

\newenvironment{proof}[1][证明]%
  {\par\noindent\normalfont{\hei\color{blue} #1.} \upshape}
  {\mbox{}\hfill\scalebox{1.2}{\ensuremath{\Box}}\medskip}

\linespread{1.3}
\setlength{\parskip}{1ex}

%%%%% ===== 自定义命令 =====================================================
\newcommand{\bbm}{\begin{bmatrix}}
\newcommand{\ebm}{\end{bmatrix}}
\newcommand{\beq}{\begin{equation}}
\newcommand{\eeq}{\end{equation}}
\renewcommand{\C}{\mathbb{C}}
\newcommand{\R}{\mathbb{R}}
\newcommand{\Rn}{\mathbb{R}^{n\times n}}
\newcommand{\IA}{\mathcal{A}}
\newcommand{\IK}{\mathcal{K}}
\newcommand{\IO}{\mathcal{O}}
\newcommand{\lam}{\lambda}
\newcommand{\eps}{\varepsilon}
\newcommand{\dis}{\displaystyle}
\newcommand{\mycite}[1]{\textcolor{blue!50!white}{\upshape{#1}}}
\newcommand{\Der}{\,\mathrm{D}}
\newcommand{\der}{\,\mathrm{d}}
%
\newcommand{\myem}[1]{\textcolor{blue}{#1}}
%

\begin{document}

\title[短标题]{学术报告标题\\ 长标题可以强制换行}
% \subtitle{也可以有个副标题}

\author[报告人姓名]{报告人姓名}

\institute[Math.ECNU]{\zihao{5} 华东师范大学~数学系}

\date[2016.05]{2016年5月}

\begin{frame}[plain]
  \titlepage
\end{frame}

\begin{frame}
  \frametitle{内容提要}
  \tableofcontents[hideallsubsections]
\end{frame}

\section{背景介绍}

% 在每节前插入目录
\AtBeginSection[]{\frame{\tableofcontents[currentsection,hideallsubsections]}}

\begin{frame}
  \frametitle{背景介绍}
\begin{myitem}
\item 考虑问题
    $$ a^2+b^2=c^2.$$

\bigskip
\item 问题应用背景
    \begin{subitem}
        \item  xxxxx
        \item  xxxx
        \item  xxxxx
        \item $\cdots\ \cdots$
    \end{subitem}
\end{myitem}
\end{frame}

\section{定义与定理}
\begin{frame}{定义与定理}

  \begin{definition}
    这是连续的定义, 这是连续的定义, 这是连续的定义,
    这是连续的定义, 这是连续的定义, 这是连续的定义,
    这是连续的定义, 这是连续的定义, 这是连续的定义.
  \end{definition}
\end{frame}

\begin{frame}{定义与定理 (续)}
  \begin{theorem}[中值定理]
    这是中值定理, 这是中值定理, 这是中值定理, 这是中值定理,
    这是中值定理, 这是中值定理, 这是中值定理, 这是中值定理,
    这是中值定理, 这是中值定理, 这是中值定理, 这是中值定理,
    这是中值定理, 这是中值定理, 这是中值定理, 这是中值定理.
  \end{theorem}
\end{frame}

\section{算法描述}
\begin{frame}{算法描述}
\begin{myitem}
  \item 基本思想
  \begin{subitem}
    \item xxxxxx
    \item xxxxxx
  \end{subitem}
  \bigskip

  \item 主要优点
  \begin{subitem}
    \item  xxxx
    \item  xxxxx
  \end{subitem}
\end{myitem}
\end{frame}


\begin{frame}{算法 1}

  算法 1

\end{frame}


\section{数值实验}

\begin{frame}{数值算例}

\begin{center}
{Numerical results for Example 1}\smallskip

\begin{tabular}{ccccccccccc} \toprule
    &&&\multicolumn{2}{c}{GMRES(C)}
    &&\multicolumn{2}{c}{GMRES(L)}
    &&\multicolumn{2}{c}{GMRES(P)} \\ \cmidrule{4-5}\cmidrule{7-8}\cmidrule{10-11}
 $\theta$ & $N$ && Iter & CPU && Iter & CPU && Iter & CPU \\\hline
  0.5
 & $2^{11}$ &&  33&    0.04 &&  13&    0.02 &&  12&   0.01 \\
 & $2^{12}$ &&  33&    0.12 &&  14&    0.04 &&  12&   0.03 \\
 & $2^{13}$ &&  33&    0.26 &&  14&    0.09 &&  12&   0.08 \\
 & $2^{14}$ &&  33&    0.53 &&  15&    0.19 &&  12&   0.15 \\ \midrule
  0.8
 & $2^{11}$ &&  33&    0.04 &&  13&    0.02 &&  12&   0.01 \\
 & $2^{12}$ &&  33&    0.11 &&  14&    0.04 &&  12&   0.03 \\
 & $2^{13}$ &&  33&    0.25 &&  15&    0.10 &&  12&   0.08 \\
 & $2^{14}$ &&  33&    0.53 &&  16&    0.21 &&  12&   0.15 \\ \bottomrule
\end{tabular}
\end{center}

\end{frame}

\section{结论与展望}
\begin{frame}{结论与展望}

   这里是结论与展望 conclusion 和 remarks

\end{frame}


\begin{frame}[c,plain]
\begin{center}
\Huge\color{red}\heiti\bfseries 谢\quad 谢!

  Thank you!
\end{center}
\end{frame}

\end{document} 