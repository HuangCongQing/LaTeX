\documentclass{aase}
\usepackage{multicol}
\usepackage{psfig}
%\usepackage{subfigure}
\usepackage{amsmath}
\usepackage{amssymb}
\usepackage{amsfonts}
\usepackage{graphicx}
\usepackage{url}
\usepackage{ccaption}
\usepackage{booktabs} % �����߱�����������������
%\usepackage{graphicx}
%\usepackage{epstopdf}
%\usepackage{epsfig}
%\usepackage{algorithm}
%\usepackage{algorithmic}
%\usepackage{color}



\makeatletter
\newcommand{\rmnum}[1]{\romannumeral #1}
\newcommand{\Rmnum}[1]{\expandafter\@slowromancap\romannumeral #1@}
\makeatother

\setcounter{page}{1225}

\firstheadname{ACTA AUTOMATICA SINICA} % ��ҳҳü
\firstfootname{} % ��ҳҳ�� \zihao{5-}$\copyright$ 2011 by {\sl Acta Automatica Sinica}. All rights reserved.
\headevenname{ACTA AUTOMATICA SINICA} % ����ҳż��ҳҳü
\headoddname{ XU Zong-Ben et al.:  Representative of $L_{1/2}$
Regularization among  $\cdots$}
% ����ҳ����ҳҳü
\begin{document}


\title{{ Representative of
\\\vskip 0.3\baselineskip
${\pmb L}_{{\bf  1{\pmb /}2}}$ Regularization among
\\\vskip 0.3\baselineskip ${\pmb {L_q}}~ {{\pmb(}{\bf 0}}\,{\pmb <}\,
{\pmb q}\,{\pmb  \leq}\, {\bf 1}{\pmb )}$ Regularizations:
\\\vskip 0.3\baselineskip \mbox{an~Experimental~Study~Based~on}\\\vskip 0.3\baselineskip  Phase
Diagram}
\thanks{
Manuscript received March 16, 2011; accepted  June 22, 2011 }
\thanks{
Supported by National Basic Research Program of China (973 Program)
(2007CB311002) and National Natural Science Foundation of China
(60975036) }
\thanks{Recommended by Associate Editor LIU Yi-Jun }
\thanks{
1. Institute of Information and System Science, Xi$'$an Jiaotong
University, Xi$'$an 710049, P.\,R.\,China \quad 2. Department of
Mathematics, Northwest University, Xi$'$an 710069, P.\,R.\,China } }

\author{
XU Zong-Ben$^1$ \hspace{1em}\ GUO Hai-Liang$^1$ \hspace{1em}\ WANG
Yao$^1$
\\ \small
ZHANG Hai$^{1,\,2}$ }


\abstract{Recently, regularization methods have attracted increasing
attention. $L_q$ $(0 < q < 1)$ regularizations were proposed after
$L_1$ regularization for better solution of sparsity problems. A
natural question is which is the best choice among $L_q$
regularizations with all $q$ in $(0,1)$? By taking phase diagram
studies with a set of experiments implemented on signal recovery and
error correction problems, we show the following: 1) As the value of
$q$ decreases, the $L_q$ regularization generates sparser solution.
2) When $1/2\leq q < 1$, the $L_{1/2}$ regularization always yields
the best sparse solution and when $0 < q \leq 1/2$, the performance
of the regularizatons takes no significant difference. Accordingly,
we conclude that the $L_{1/2}$ regularization can be taken as a
representative of $L_q$ $(0 < q < 1)$ regularizations.}

\keyword{$L_q$ regularization, phase diagram, signal recovery, error
correction}

\doi{10.3724/SP.J.1004.2012.01225}

\cit{Xu Zong-Ben, Guo Hai-Liang, Wang Yao, Zhang Hai. Representative
of $L_{ 1 /2}$ regularization among
 $L_q$ $( 0 < q\leq 1 )$ regularizations:
an experimental study based on  phase diagram. \textsl{Acta
Automatica Sinica}, 2012, \textbf{38}(7): 1225$-$1228}

\maketitle

\pagestyle{aasheadings}



Recently, considerable attention has been paid to the following
sparsity problem. We are given an $n \times N$ matrix $\Phi$ which
is in some sense ``random'', for example, a matrix with i.i.d
Gaussian entries, and we are also given an $n$-vector $\pmb{y}$ and
know that $\pmb{y}=\Phi \pmb{x}_{0}$ where
$\pmb{x}_{0}\in\textbf{R}^N$ is an unknown sparse vector. We expect
to recover $\pmb{x}_{0}$. However, $n$ $\ll$ $N$, the system of
equations is underdetermined and hence, it is not a properly-posed
problem in linear algebra. Nevertheless, sparsity of $x_{0}$ is a
very useful priority that sometimes allows unique solution.
Accordingly, one naturally proposes to use the following
optimization model ($P_0$) to obtain the sparsest solutions
$$
(P_0)\ \ \ \ \min_{\pmb{x}\in\textbf{R}^N}\|\pmb{x}\|_0 \ \
\mbox{s.\,t.}\ \pmb{y} = \Phi \pmb{x} \eqno(1)
$$
where $\|\pmb{x}\|_0$ = $|\{i:x_i\neq0\}|$. This is of little
practical use, however, since the problem $(P_0)$ is combinatorial
in feature and generally difficult to be solved as its solution
requires an intractable combinatorial search$^{[1]}$.



To solve this problem, the subsequent $(P_1)$ optimization problem
was suggested$^{[2]}$, which then can be transformed into a linear
programming problem:
$$
(P_1)\ \ \ \ \min_{\pmb{x}\in\textbf{R}^N}\|\pmb{x}\|_1 $$
$$\mbox{s.\,t.}\ \qquad \pmb{y}=\Phi \pmb{x} \eqno(2)
$$
where $\|\pmb{x}\|_1$ = $\sum_{i=1}^N|x_i|$. We call ($P_0$) as
$L_0$ regularization and ($P_1$) as $L_1$ regularization. The use of
$L_1$ regularization has become so widespread that it has been
arguably considered as the ``modern least squares''$^{[3]}$.
However, the solutions of the $L_1$ regularization are often not as
sparse as those of the $L_0$ regularization. To find  solutions more
sparse than $L_1$ regularization is definitely imperative and
required for many applications. A natural try for this purpose is to
apply the $L_q$ (0 $<$ $q$ $<$ 1) regularization, that is, to solve
the following ($P_q$) model,



$$
(P_q)\ \ \ \  \ \  \min_{\pmb{x}\in\textbf{R}^{N}} \|\pmb{x}\|_q \
$$

$$\mbox{s.t.}\qquad \pmb{y} =\Phi \pmb{x} \eqno(3)
$$
or equivalently,
$$
\min_{\pmb{x}\in\textbf{R}^{N}}\{\|\pmb{y}-\Phi \pmb{x}\|_2^2+
\lambda \|\pmb{x}\|_q^q\} \eqno(4)
$$
where $\|\pmb{x}\|_q$ = $\sum_{i=1}^N|x_i|_q^q$, and $\lambda$ is a
regularization parameter. Obviously, the $L_q\,( 0 < q < 1)$ model
is no longer a convex optimization problem, and thus we can only get
the local optimal solutions in most cases, yet it can yield
solutions sparser than the $L_1$ regularization model$^{[4-5]}$.

There are many choices for $q$ when the $(P_q)$ model is adopted. A
nature and also crucial question is: which $q$ is the best among
$L_q$ $(0 < q < 1)$ regularizations? In this paper, our aim is to
provide an affirmative answer to this question through an
experimental study with phase diagram. We will comparatively apply
the $L_q$ $( 0 < q \leq 1)$ regularizations, according to the phase
diagram requirement, to several typical sparsity problems:
compressive sensing and error correction, and then, we will conclude
from the resultant phase diagrams that $L_{1/2}$ regularization can
be taken as a representative of $L_q$ $(0 < q < 1)$ regularizations.
This study  offers a solid evidence to support the speciality and
importance of $L_{1/2}$ regularization.

%-------------------------------------------------------------------------

\section{Experimental methods and test problems}

%-------------------------------------------------------------------------

\subsection{Experimental methods}







For an underdetermined system of linear equations $\pmb{y}=\Phi
\pmb{x}$, when the model ($P_0$) has a unique sparse solution (it is
then also the unique solution of ($P_q$)) and the solution can be
obtained from the $L_q$ $(0 < q \leq 1)$ regularization procedure,
we say that the $L_q$ and $L_0$ regularizations are equivalent, or
briefly, of $L_0/L_q$ equivalence. When a vector is not only a
solution of ($P_0$) but also a solution of ($L_q$) problem, it is
said to be a point of $L_0/L_q$ equivalence.

In the context of $L_0/L_1$ equivalence, Donoho$^{[6-7]}$ introduced
the notion of phase diagram to illustrate how sparsity (number of
nonzeros in $\pmb{x}$/number of rows in $\Phi$) and indeterminacy
(number of rows in $\Phi$/number of columns in $\Phi$) affect the
success of $L_1$ regularization. Using the technique of
high-dimensional geometry analysis, Donoho$^{[6]}$ provided a
necessary and sufficient condition on a particular random matrix
$\Phi$ of size $n \times N$ such that every $\pmb{x}$ $\in$
$\chi^N(k)$ is a point of $L_1$/$L_0$ equivalence, where $\chi^N(k)$
= $\{\pmb{x}$ $\in$ ${\bf R}^N: \|\pmb{x}\|_0 \leq k \}$. The
performance exhibits two phases (success/failure) in a diagram, as
shown in Fig.\,1$^{[7]}$. Each point on the plot of the figure
corresponds to a statistical model for certain values of $n$, $N$,
and $k$. The abscissa runs from 0 to 1, and gives values for $\delta
= n/N$. The ordinate is $\rho = k/n$, measuring the level of
sparsity in the model. Above the plotted phase transition curve, the
$L_1$ method fails to find the sparsest solution; below the curve,
the solution of ($P_1$) is the precise solution of ($P_0$).

 \vskip3mm {\centering
\vbox{\centerline{\psfig{figure=pic1.eps}} \vskip2mm {\small
Fig.\,1\quad Theoretical phase transition diagram: the theoretical
threshold at which equivalence of the solutions to the $L_1$ and
$L_0$ optimization problems breaks down (Along the $x$-axis the
level of underdeterminedness decreases, and along the $y$-axis the
level of sparsity of the underlying model increases.) }}}\vskip3mm

Donoho et al.$^{[8]}$ conducted a series of simulation experiments
for the problem of variable selection when the number of variables
exceeds the number of observations. They have defined a problem
suite $S\left\{k,n,N\right\}$ as a collection of problems with
sparse solutions, and each problem has an $n$ $\times$ $N$ model
matrix $\Phi$ and a $k$-sparse $N$-vector of coefficients $\pmb{x}$.
We will follow the method of Donoho et al.$^{[8]}$ in this paper.
For each $k$, $n$, $N$ combination we run an algorithm in question
multiple times, and measure its success according to a quantitative
criterion. We choose the relative root square error (RRSE) RRSE =
$\|{\hat{\pmb{x}}}-{\pmb x}\|_2/\|\pmb{x}\|_2$ as the quantization
criterion, and the results are then compared across models with
different problem sizes.

The following recipes are employed to study an algorithm for $L_q$
$(0 < q \leq 1)$ regularization models:

1) Generate a prototype model $\pmb{y} = \Phi \pmb{x}$, which has
a $k$-sparse solution, where $k < N$.

2) Run an algorithm of $L_q$ regularization to obtain a
reconstructed solution $\hat{\pmb{x}}$.

3) Evaluate performance to test if ${\|{\hat{\pmb{x}}}-{\pmb
x}\|_2}/{\|\pmb{x}\|_2} \leq \gamma$, where $\gamma$ is a
tolerance bound set in advance.


After getting all the RRSEs of the problem suite $S\{k,n,N\}$, a
phase diagram can be drew for the tasted algorithm. With such
methodology, we will present the experiments in the next section.
However, we need to first introduce two typical test problems.

%-------------------------------------------------------------------------
\subsection{Test problems}
%-------------------------------------------------------------------------

\subsubsection{Signal recovery}

Assume that $\pmb{x}$ is a signal with $k$ nozero spikes, and that
$\Psi$ is a unit matrix, i.e.,  the canonical basis is used to
denote the signal. We attempt to reconstruct $\pmb{x}$ through
random sampling $\Phi$ by ($P_1$) and ($P_q$) models and compare
the performance of $L_q$ regularizations when $q$ varies from 0 to
1.


Many researchers have suggested methods to solve the $L_q$ $(0 < q
\leq 1)$ regularization problems. Since $L_1$ regularization is
convex and the others are not, there are many exclusive efficient
methods for $L_1$ regularization, while very few for $L_q$ $(0 < q <
1)$ regularizations. We briefly introduce the methods  applied in
the experiments below.

For the $L_1$ regularization problem, we will use the basis
pursuit method suggested by Donoho et al.$^{[2]}$, which is based
on a linear program solver.

For the $L_q$ regularization problems, we will apply the reweighted
$L_1$ method proposed by Xu et al.$^{[9]}$, which transforms the
$L_q$ problems into a series of convex weighted $L_1$ regularization
problems, to which the existing $L_1$ regularization algorithms can
be efficiently applied$^{[10-11]}$.
%------------------------------------------------------------------------
\subsubsection{Error correction}

Suppose that a response variable $\pmb{y}$ is dependent on
independent variables $A_1,\cdots, A_p$, and that we are in a
classically designed experiment, with $p < n$. If the dependence
is linear, we then have $\pmb{y} = \sum_j x_j A_j + \pmb{e}$,
where the error $\pmb{e}$ is a ``wild'' variable containing
occasionally very large outliers. We assume that the outlier
generators $\pmb{e}$ have most entries of 0, with $k$ being the
number of nozero entries, but we know neither  which entries are
affected nor how they are affected. We would like to recover the
information $\pmb{x}$ exactly from the corrupted $N$-dimensional
vector $\pmb{y}$.

To decode, Candes et al.$^{[12]}$ proposed to use the subsequent
($D_1$) model to solve the error correction problem:
$$
\mbox{($D_1$)}\ \ \
\min_{\tilde{\pmb{x}}\in\textbf{R}^p}\|\pmb{y}-A\tilde{\pmb{x}}\|_{1}\eqno(5)
$$
which can also be recast as an linear programming (LP) problem. They
showed that if $A$ is chosen suitably, the solution of ($D_1$) model
can correctly retrieve the information $\pmb{x}$ without error
provided that $\pmb{e}$ is sparse enough.

Also, Chartrand$^{[13]}$ suggested to solve the error correction
problem by $L_q$ $(0 < q < 1)$ minimization:
$$
\mbox{($D_q$)}\ \ \
\min_{\tilde{\pmb{x}}\in\textbf{R}^p}\|\pmb{y}-A\tilde{\pmb{x}}\|_{q}\eqno(6)
$$

For the $L_1$ minimization, we use the corresponding Matlab program
in $L_1$-magic software, which is available on Candes$'$ web
page$^{[14]}$. For $L_q$ $(0 < q < 1)$ minimization, we slightly
adjust the reweighted $L_1$ algorithm proposed in [9] to solve the
$(D_q)$ model.

%-------------------------------------------------------------------------

\section{Experiment results}

%-------------------------------------------------------------------------
\subsection{Signal recovery}






To apply the phase diagram methodology, we fix the length of
signal, $N = 512$, and then build a prototype model for a certain
level of underdeterminedness $\delta \equiv  n/N$ for $\delta \in
[0, 1]$, and a sparsity level $\rho \equiv  k/n$ for $\rho \in [0,
1]$. Then, a problem suite is constructed through varying sampling
number $n$ and sparsity $k$. The performance of the regularization
algorithms is then evaluated over this grid in a systematic way.
For each $k,n,N$ combination, the procedure of study is as
follows:

{\bf  Algorithm 1.}

{\bf Step 1.} Generate $A_{n\times N}$ with $A_{ij} \in {\rm N}(0,
1)$, and create $\pmb{y}$ $=$ $A\pmb{x}$ where $\pmb{x}$ has $k$
nonzeros drawn from ${\rm N}(0,1)$.

{\bf Step 2.} Run a regularization method, and get the reconstructed
solution $\hat{\pmb{x}}$.

 {\bf Step 3.} Evaluate
the ``success/failure'' property: if the relative root square error
(RRSE) is smaller than $10^{-5}$, the recovery is considered
success, or failure otherwise.

{\bf Step 4.} Repeat Steps 1\,$\sim$\,3 for 50 times, and evaluate
the frequency of success/failure recovery.


After getting the ``success/failure frequency'' of all the defined
problem suite $S\{k,n,N\}$, we can plot it on the phase plane
$(\delta, \rho)$, where $\delta =  n/p$ and $\rho =  k/n$ (Fig.\,2).
The contours indicate the success rate, where light gray (above the
belt curve) means the success rate of this combination of
$\{k,n,N\}$ is 0\,\%, and dark gray (below the belt curve) means the
success rate of this combination of $\{k,n,N\}$ is 100\,\%. We can
also find the belt area with other color in Fig.\,2, which means
that the success rate is between 0\,\% and 100\,\%.  Fig.\,2 shows
the recovery performances of $L_{0.1}$, $L_{0.3}$, $L_{0.5}$,
$L_{0.7}$, $L_{0.9}$, and $\ L_{1.0}$ separately, in which the thin
curve is the theoretical $L_1/L_0$ equivalent curve, proved by
Donoho$^{[6-7]}$. After comparing the phase diagrams in Fig.\,2, we
can obtain the following conclusions:

1)  The phase transition phenomenon occurs very clearly, with the
belt area displaying a phase transition curve for any $L_q$
regularizations.

2)  The $L_q$ phase transition curves are almost all above the
theoretical $L_1/L_0$ equivalent curve, showing that $L_q$
regularizations have stronger sparsity promoting ability   than
$L_1$ regularization.

3)  The smaller $q$, the better the performance, but when $q$ $\in$
$(0,0.5)$, the difference is invisible.


 \vskip4mm {\centering
\vbox{\centerline{\psfig{figure=pic2.eps}} \vskip2mm {\small
Fig.\,2\quad Phase diagrams of $L_q$ regularizations ($q$ = 0.1,
0.3, 0.5, 0.7, 0.9,
  1.0) when applied to signal recovery, where abscissa $\delta =
  n/N$ and ordinate $\rho =  k/n$}}}\vskip4mm





To quantitatively compare the differences of behaviors of the $L_q$
regularizations for different $q\in(0,1)$, we have gotten more phase
diagrams for different values of $q$ and calculated the success
percentage in all the suite for a regularization, that is, the ratio
of the dark gray region in the whole region of the phase plane. The
interpolated success percentage curve is depicted in Fig.\,3, where
the horizontal axis is the value of $q$, and the vertical axis is
the percentage of successive restoration. From Fig.\,3, we can
observe the following:


1)  The ratio changes very slowly between $q \in (0,0.5)$.

 \vskip4mm {\centering
\vbox{\centerline{\psfig{figure=pic3.eps}} \vskip2mm {\small
Fig.\,3\quad Success recovery percentage of  $L_q$ regularizations
when
  applied to signal recovery}}}\vskip3mm



2)  The ratio changes rapidly when $q$ increases from 0.5 to 1.

This reveals that $L_{1/2}$ regularization is significantly better
than $L_1$ regularization, while it takes no significant
difference from other $L_q$ regularizations when $q\in(0,0.5)$.
Thus, in the sense of bringing the benefit of exact recovery,
$L_{1/2}$ can be regarded as the best.
%-------------------------------------------------------------------------
\subsection{Error correction}

Inspired by the phase transition experiment for signal recovery, we
fix $n$ = 512, and then build a prototype model for a fixed $\gamma
= p / n $, $\gamma \in [0,1]$, and a fixed $\epsilon = k / p$ for
$\epsilon$ $\in$ $[0,1]$. Almost exactly the same with the
experiment of signal recovery, for each $k, p, n$ combination, the
procedure of study is  given as follows.

{\bf Algorithm 2.}

{\bf Step 1.} Generate $A=(A_{ij})\in \textbf{R}^{n\times p}$ with
$A_{ij} \thicksim {\rm N}(0, 1)$ and $p$-dimensional vector
$\pmb{x_0}$ with $x_i \thicksim {\rm N}(0, 1)$, and then create
$\pmb{y} = A\pmb{x_0}+\pmb{e}$ where $\pmb{e}$ has $k$ nonzeros
drawn from ${\rm N}(0,1)$.

{\bf Step 2.} Run a regularization method, and get the solution
$\hat{\pmb{x}}$.

{\bf Step 3.} Evaluate the ``success/failure'' property: if the
relative root square error (RRSE) is smaller than $10^{-5}$, the
recovery is considered success, or failure otherwise.

{\bf Step 4.} Repeat Steps 1\,$\sim$\,3 for 50 times, and evaluate
the ``success/failure''.







After getting the ``success/failure'' of all the defined problem
suite $S\{k,p,n\}$, we plot it on the plane $(\gamma, \epsilon)$,
where $\gamma = p/n$ and $\epsilon =  k/p$. The contours indicate
the success/failure rate, with light gray (above the belt curve)
meaning the success of this combination of $\{k,n,p\}$ is 0\,\%, and
dark gray (below the belt curve) means the success rate of this
combination of $\{k,n,p\}$ is 100\,\%. In the figure, we can also
find the belt area with other color, which means that the success
rate is between 0\,\% and 100\,\%. Following Donoho and
Tanner$^{[15]}$, we display the result in different ordinate systems
with variables $\rho = k / (n-p)$ and $\delta = (n - p )/n$. Fig.\,4
then shows the performances of $L_{0.1},\ L_{0.3},\ L_{0.5},\
L_{0.7},\ L_{0.9}$, $\ L_{1.0}$ regularization algorithms, in which
the thin curve is the theoretical $L_1/L_0$ equivalent curve. After
comparing the phase diagrams in Fig.\,4, we find that all the
phenomena observed in signal recovery application occur again in
this error correction application.

 \vskip3mm {\centering
\vbox{\centerline{\psfig{figure=pic4.eps}} \vskip2mm {\small
Fig.\,4\quad Phase diagrams of $L_q$ regularizations ($q$ = 0.1,
0.3, 0.5, 0.7, 0.9,
  1.0) when applied to error correction, where abscissa $\delta =
  {(n-p)}/n$ and ordinate $\rho =  k/{(n-p)}$}}}\vskip3mm


To quantitatively compare the differences of behaviors of the $L_q$
minimizations for different $q\in(0,1)$, we have also gotten more
phase diagrams for different values of $q$ and calculated the
success percentage in all the suite for a regularization algorithm,
which is defined as the ratio of the dark gray region in the whole
phase plane. The interpolated success rate curve is given in
Fig.\,5. These experiments for error correction reveal again that
$L_{1/2}$ regularization is significantly better than $L_1$
regularization, while it takes no significant difference from other
$L_q$ regularization when $q\in(0,0.5)$. Thus, in the sense of
getting the benefit of exact recovery, $L_{1/2}$ is the best.


 \vskip3mm {\centering
\vbox{\centerline{\psfig{figure=pic5.eps}} \vskip2mm {\small
Fig.\,5\quad Success recovery percentage of  $L_q$ regularizations
when
  applied to error correction}}}\vskip3mm

\section{Conclusion}

With the phase diagram tool, we have conducted an experimental study
on performance comparison between $L_1$ regularization and $L_q$ $(0
< q < 1)$ regularizations for sparse signal recovery and error
correction. The comparisons show that when $0 < q < 1$, the $L_q$
regularizations can always yield sparser solutions than $L_1$
regularization, and that the smaller, the better the performance of
$L_q$ regularizations. Nevertheless, when $0 < q \leq 1/2$, the
performance of $L_q$ regularizations has no significant difference.
This suggests that among the $L_q$ regularizations with $0 < q \leq
1$, $L_{1/2}$ can be taken as a representative.






The study of this paper reveals the extreme importance and special
role of $L_{1/2}$ regularization. It particularly leads to a guess
or an expectation that the $L_{1/2}$ regularization might be more
powerfully applied to sparsity problems. In a very recent
research,
 a fast efficient iterative half
thresholding algorithm was suggested for implementation of
$L_{1/2}$ regularization.



\begin{thebibliography}{99}
\zihao{6} \addtolength{\itemsep}{-0.0em} \urlstyle{rm}

\bibitem{1} Natarajan B K. Sparse approximate solutions to linear systems. \textsl{SIAM Journal on Computing}, 1995, \textbf{24}(2): 227$-$234
\bibitem{2} Chen S, Donoho D L, Saunders M A. Atomic decomposition by basis pursuit. \textsl{SIAM Journal on Scientific Computing}, 1998, \textbf{20}(1): 33$-$61
\bibitem{3} Candes E, Wakin M, Boyd S. Enhancing sparsity by reweighted $L_{1}$ minimization. \textsl{Journal of Fourier Analysis and Applications}, 2008, \textbf{14}(5): 877$-$905
\bibitem{4} Chartrand R. Exact reconstructions of sparse signals via nonconvex minimization. \textsl{IEEE Signal Processing Letters}, 2007, \textbf{14}(10): 707$-$710
\bibitem{5} Chartrand R, Staneva V. Restricted isometry properties and nonconvex compressive sensing. \textsl{Inverse Problems}, 2008, \textbf{24}(3): 1$-$14
\bibitem{6} Donoho D L. Neighborly polytopes and the sparse solution of underdetermined linear equations [Online], available: http: // www-stat.stanford.edu/ $\sim $donoho/ Reports/ 2005/ NPaSSULE-01-28-05.pdf, November 8, 2011
\bibitem{7} Donoho D L. High-dimensional centrally symmetric polytopes with neighborliness proportional to dimension. \textsl{Discrete and Computational Geometry}, 2006, \textbf{35}(4): 617$-$652
\bibitem{8} Donoho D L, Stodden V. Breakdown point of model selection when the number of variables exceeds the number of observations. In: Proceedings of the International Joint Conference on Neural Networks. Vancouver, USA: IEEE, 2006. 1916$-$ 1921
\bibitem{9} Xu Z B, Zhang H, Wang Y, Chang X Y, Yong L. $L_{1/2}$ regularization. \textsl{Science in China Series F: Information Sciences}, 2010, 53(6): 1159$-$1169
\bibitem{10} Chen X, Xu F M, Ye Y. Lower bound theory of nonzero entries in solutions of $L_{2}$-$L_{p}$ minimization. \textsl{SIAM Journal on Scientific Computing}, 2010, \textbf{32}(5): 2832$-$2852
\bibitem{11} Chartrand R, Yin W. Iteratively reweighted algorithms for compressive sensing. In: Proceedings of the IEEE International Conference on Acoustics, Speech and Signal Processing. Las Vegas, USA: IEEE, 2008. 3869$-$3872
\bibitem{12} Candes E, Tao T. Decoding by linear programming. \textsl{IEEE Transactions on Information Theory}, 2005, \textbf{51}(12): 4203$-$ 4215
\bibitem{13} Chartrand R. Nonconvex compressed sensing and error correction. In: Proceedings of the IEEE International Conference on Acoustics, Speech and Signal Processing. Honolulu, USA: IEEE, 2007. 889$-$892
\bibitem{14} Candes E, Caltech J R. $L_{1}$-MAGIC: recovery of sparse signals via convex programming [Online], available: http: //users.ece.gatech.edu/$^\sim $justin/l1magic/downloads/l1magic. pdf, June 30, 2011
\bibitem{15} Donoho D L, Tanner J. Observed universality of phase transitions in high-dimensional geometry, with implications for modern data analysis and signal processing. \textsl{Philosophical Transactions of the Royal Society A}, 2009, \textbf{367}(1906): 4273 $-$4293



\end{thebibliography}

\begin{biographynophoto}
\noindent{\bf XU Zong-Ben }\quad Professor at the Institute for
Information and System Sciences, Faculty of Science, Xi$'$an
Jiaotong University. His research interest covers machine
learning, visual computation and image processing, and sparse
modeling and optimization. E-mail: zbxu@mail.xjtu.edu.cn
\end{biographynophoto}

\begin{biographynophoto}
\noindent{\bf GUO Hai-Liang }\quad Master student at the Institute
for Information and System Sciences, Faculty of Science, Xi$'$an
Jiaotong University. His research interest covers compressive
sensing, sparse modeling of signals and images. Corresponding
author of this paper. E-mail: nianlongguo@gmail.com
\end{biographynophoto}

\begin{biographynophoto}
\noindent{\bf WANG Yao }\quad Ph.\,D. candidate at the Institute for
Information and System Sciences, Faculty of Science, Xi$'$an
Jiaotong University. His research interest covers gradient boosting
and its application to variable selection, compressive sensing, and
sparse modeling of signals and images.\\E-mail: yao.s.wang@gmail.com
\end{biographynophoto}

\begin{biographynophoto}
\noindent{\bf ZHANG Hai }\quad Associate professor in the Department
of Mathematics, Northwest University and  Ph.\,D. candidate at the
Institute for Information and System Sciences, Faculty of Science,
Xi$'$an Jiaotong University. His research interest covers stability
of learning algorithms, learning theory, sparsity and
regularization. E-mail: zhanghaiwu@tom.com
\end{biographynophoto}
%\end{multicols}
\end{document}
